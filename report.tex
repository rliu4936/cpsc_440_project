\documentclass{article}
\usepackage{graphicx} % <-- Add

\usepackage{project_440_550}
% Please submit it as is here, with line numbers.
% If you'd like a "less draft"-looking version for your website or something after:
%     \usepackage[final]{project_440_550}   % keeps the footer but kills line numbers,
%     \usepackage[preprint]{project_440_550}   % removes both


\usepackage[utf8]{inputenc} % allow utf-8 input
\usepackage[T1]{fontenc}    % use 8-bit T1 fonts

\usepackage[USenglish]{babel}  % there's a "canadian" option, but it's an alias for USenglish,
                               % and for some reason it makes csqoutes behave differently...
\usepackage{csquotes}       % smarter handling of quotes (used by biblatex)

\usepackage{booktabs}       % professional-quality tables
\usepackage{amsfonts}       % blackboard math symbols
\usepackage{nicefrac}       % compact symbols for 1/2, etc.
\usepackage{microtype}      % microtypography
\usepackage{xcolor}         % colors
\usepackage{hyperref}       % hyperlinks
\usepackage{url}            % simple URL typesetting

% I recommend the biblatex package.
% If you hate it for some reason, though, you can use natbib instead:
% comment out this block and uncomment the next one.

\usepackage[style=authoryear,maxbibnames=30,natbib]{biblatex}
\addbibresource{refs.bib}
\renewbibmacro{in:}{}  % drops a silly "In:" from biblatex format
\DeclareDelimFormat{nameyeardelim}{\addspace} % remove a comma i dislike, doesn't matter
\DeclareNameAlias{sortname}{given-family}  % avoid kinda-weird bib printing order

%\usepackage[round]{natbib}
%\newcommand{\printbibliography}{\bibliographystyle{plainnat}\bibliography{refs}}


\usepackage[capitalize,noabbrev]{cleveref}


\title{Technical Indicator and Machine Learning-Based Trading Strategies}


% The \author macro works with any number of authors. There are two commands
% used to separate the names and addresses of multiple authors: \And and \AND.
%
% Using \And between authors leaves it to LaTeX to determine where to break the
% lines. Using \AND forces a line break at that point. So, if LaTeX puts 3 of 4
% authors names on the first line, and the last on the second line, try using
% \AND instead of \And before the third author name.

\author{%
  Ronald Liu\\
  \texttt{rliu4936@student.ubc.ca}
}

\begin{document}
\maketitle


\begin{abstract}
    This project explores technical indicator-driven and machine learning-based trading strategies, focusing on both individual stock indices and a broad equity universe to evaluate strategy robustness and generalizability.
\end{abstract}


\section{Introduction and Motivation}

This project focuses on modeling daily price movements instead of finer intraday data. This choice helps manage data volume and avoids complexities associated with high-frequency trading noise.

We initially explored a single stock index, but found it overly simplistic. With only about a thousand data points, even a basic two-parameter technical indicator could capture most major movements. This left little room for effective machine learning modeling and increased the risk of overfitting.

To address this, we shift our focus to technical indicators across the 500 stocks within the S\&P 500. This larger and more diverse dataset makes overfitting significantly harder.

To better understand the behavior of simple technical indicators, we visualize the moving average crossovers and log returns below.

\begin{figure}[h]
    \centering
    \includegraphics[width=\textwidth]{MA_cross_level.png}
    \caption{Figure 1 illustrates the moving average crossover strategy, highlighting intersections between short-term and long-term averages that signal potential trade opportunities.}
    \label{fig:ma_cross_level}
\end{figure}

\begin{figure}[h]
    \centering
    \includegraphics[width=\textwidth]{strategy_vs_buyhold_with_markers_QQQ_2000-01-01_to_2025-04-26.png}
    \caption{Figure 2 presents the log returns over time, emphasizing the volatility and distribution of returns in the dataset.}
    \label{fig:log_returns}
\end{figure}

Figure 1 highlights key crossover points between short-term and long-term moving averages, signaling potential trades. Figure 2 depicts the log returns over time, illustrating the volatility inherent in the dataset. Using the moving average crossover strategy, we achieved a return of 1655.00\%. However, this strong performance on a single index raises concerns about potential overfitting.

To provide further context, the moving average crossover strategy on the QQQ index achieved a total return of 1655.00\% over the period from 2000 to 2025, significantly outperforming the buy-and-hold return of 488.39\%. The best-performing configuration involved a short moving average window of 41 days and a long moving average window of 140 days, with an annualized Sharpe ratio of 0.79, indicating strong risk-adjusted returns. These results were generated using a custom backtesting framework built in Python, leveraging \texttt{backtrader} for simulation and \texttt{yfinance} for historical data retrieval. The codebase is designed for efficient parameter search with caching mechanisms to avoid redundant computations.




% --- Stock Selection and Filtering Section ---

\section{Stock Selection and Filtering}

To increase the generalizability and robustness of our model, we decided to expand our analysis from a single stock index (QQQ) to the broader S\&P 500 index. However, a key challenge was ensuring that all stocks in the S\&P 500 had sufficient data, particularly from January 1st, 2000, to avoid issues with incomplete historical data. 

The process involved:
\begin{itemize}
    \item Downloading the tickers for the S\&P 500 from Wikipedia.
    \item Filtering the tickers to only include those with data available since the start of 2000.
    \item Storing the tickers with sufficient data in a separate file for further analysis.
\end{itemize}

After filtering, we were left with a dataset of 353 stocks that met the data availability criteria. This step helps ensure that our models are not overfitting to stocks with missing or sparse data. The resulting list of stocks was then used in our backtesting process.

The list of tickers with sufficient historical data is stored in a separate file for future reference and to avoid re-fetching data in subsequent experiments.

\begin{figure}[h]
    \centering
    \includegraphics[width=\textwidth]{filtered_stocks_over_time.png}
    \caption{Figure 3: Stocks filtered to include only those with data available from January 1st, 2000, for the backtesting process. The final list of stocks was reduced to 353.}
    \label{fig:filtered_stocks}
\end{figure}

This selection process ensures that we can evaluate the robustness of trading strategies across a diverse set of stocks while minimizing data gaps.

\section{Feature Engineering and Labeling}

After selecting stocks with sufficient historical data, we engineered a rich set of technical indicator-based features to capture trend, momentum, volatility, and volume patterns.

Initially, we constructed a set of features such as moving average crossovers, relative strength index (RSI) thresholds, and volume spikes. To further expand the feature space, we generated multiple variants of similar indicators with different window lengths and thresholds. This approach allowed us to capture different market dynamics, such as short-term versus long-term trends. We deliberately allowed feature duplication across different parameterizations, which increased feature diversity without relying on fundamentally different indicator types.

After initial experiments, we removed features that rarely triggered (those that produced a positive signal less than 5\% of the time). This pruning step ensured that our feature set remained informative and avoided introducing sparse, noisy signals. The final dataset contains over 40 well-balanced signals that vary across trend, momentum, volatility, and price action categories.

For labeling, we adopted a simple but effective rule: we labeled a day as a positive outcome (1) if the stock’s future 5-day return exceeded a chosen threshold (e.g., +2\%), and as a negative outcome (0) otherwise. This allowed us to frame our machine learning task as a binary classification problem, aiming to predict whether a stock is likely to rise substantially over the near term based on current technical indicators.

% --- Discussion Section ---
\section{Discussion}
Our analysis utilizes technical indicators as the primary features due to their computational efficiency. However, indicators that are solely based on price data tend to underperform in the long term in US stocks, as confirmed by existing literature. Relying exclusively on price-based indicators can lead to overfitting, particularly in a limited dataset like an index.

This aligns with the broader literature on technical analysis, which suggests that strategies based solely on price data often struggle to outperform in efficient markets like the US. Given the sustained long-term growth of the US stock market, the buy-and-hold strategy may be more advantageous than attempting to time the market using purely technical indicators.

To enhance the robustness of our analysis, we propose the following next steps:
\begin{enumerate}
    \item Extend our models to the entire S\&P 500 universe, incorporating additional market factors beyond price data, such as volatility and fundamentals.
    \item Evaluate performance consistency across different stocks and market conditions.
    \item Develop new features such as volatility metrics, moving averages of volume, and relative strength indicators to improve predictive power.
    \item Implement ensemble methods that combine the strengths of multiple models and indicators.
    \item Conduct statistical significance tests to ensure that the observed returns are meaningful and not due to random chance.
    \item Incorporate fundamental features such as P/E ratio and earnings growth to complement the technical analysis and improve model generalizability.
    \item Explore deeper integration of asset pricing theory to inform the feature engineering and modeling process.
\end{enumerate}

We have designed the codebase following best practices in object-oriented programming (OOP) to facilitate future extensions. We plan to explore clustering techniques for feature grouping, as well as experiment with different classifiers and labelers. Additionally, we will integrate more data from other stocks and expand the feature set to include additional market data.

Given the consistent upward trend in US markets, it is more advantageous to stay invested in the long term rather than attempting to time the market using price-based technical indicators. However, this conclusion may not hold true for markets with different behaviors, such as Japan or the UK. The model's performance and strategies will continue to evolve as we expand our analysis and explore more advanced machine learning techniques.

\section{Summary of Findings}

\textbf{Performance Summary for 30 Test Tickers}:
\begin{itemize}
    \item \textbf{Average Total Return}: 5.28\%
    \item \textbf{Average Buy-and-Hold Return}: 18.98\%
\end{itemize}

The performance results highlight that the buy-and-hold strategy significantly outperformed the model-based strategies across the 30 test tickers. While the model-generated returns were positive, they were generally smaller compared to the passive investment approach.

\printbibliography


%%%%%%%%%%%%%%%%%%%%%%%%%%%%%%%%%%%%%%%%%%%%%%%%%%%%%%%%%%%%
we 
\appendix

\section{Supplementary material} \label{app:info}

This stuff doesn't count towards your page limit.

\end{document}
